% mnras_template.tex 
%
% LaTeX template for creating an MNRAS paper
%
% v3.0 released 14 May 2015
% (version numbers match those of mnras.cls)
%
% Copyright (C) Royal Astronomical Society 2015
% Authors:
% Keith T. Smith (Royal Astronomical Society)

% Change log
%
% v3.0 May 2015
%    Renamed to match the new package name
%    Version number matches mnras.cls
%    A few minor tweaks to wording
% v1.0 September 2013
%    Beta testing only - never publicly released
%    First version: a simple (ish) template for creating an MNRAS paper

%%%%%%%%%%%%%%%%%%%%%%%%%%%%%%%%%%%%%%%%%%%%%%%%%%
% Basic setup. Most papers should leave these options alone.
\documentclass[fleqn,usenatbib]{mnras}

% MNRAS is set in Times font. If you don't have this installed (most LaTeX
% installations will be fine) or prefer the old Computer Modern fonts, comment
% out the following line
\usepackage{newtxtext,newtxmath}
% Depending on your LaTeX fonts installation, you might get better results with one of these:
%\usepackage{mathptmx}
%\usepackage{txfonts}

% Use vector fonts, so it zooms properly in on-screen viewing software
% Don't change these lines unless you know what you are doing
\usepackage[T1]{fontenc}
\usepackage{ae,aecompl}

\usepackage{graphicx}	% Including figure files
\graphicspath{ {../images/} }   % [added by JP - home, then down 1 level]


%%%%% AUTHORS - PLACE YOUR OWN PACKAGES HERE %%%%%

% Only include extra packages if you really need them. Common packages are:
\usepackage{graphicx}	% Including figure files
\usepackage{amsmath}	% Advanced maths commands
\usepackage{amssymb}	% Extra maths symbols
\usepackage{verbatim}   % added by JP for the TC word count output.


%%%%%%%%%%%%%%%%%%%%%%%%%%%%%%%%%%%%%%%%%%%%%%%%%%

%%%%% AUTHORS - PLACE YOUR OWN COMMANDS HERE %%%%%

% Please keep new commands to a minimum, and use \newcommand not \def to avoid
% overwriting existing commands. Example:
%\newcommand{\pcm}{\,cm$^{-2}$}	% per cm-squared

% Please keep new commands to a minimum, and use \newcommand not \def to avoid
% overwriting existing commands. Example:
\newcommand{\fluxdensity}{\,W m$^{-2}$}	% watts per square meter.
\newcommand{\Mpc}{\,Mpc}	% Mega parsecs.
\newcommand{\Lsun}{\,$L_{\sun}$}	% Solar luminosity.
\newcommand{\Msun}{\,$M_{\sun}$}	% Solar mass.
\newcommand{\kms}{\,km s$^{-1}$}	% kilometres per second.
\newcommand{\ms}{\,m s$^{-1}$}	    % metres per second.

%%% JP added the following:

\newcommand*\diff{\mathop{}\!\mathrm{d}} % upright d's in integrals

\newcommand{\red}[1]{{\textcolor{red}{#1}}}
\newcommand{\green}[1]{{\textcolor{green}{#1}}}
\newcommand{\blue}[1]{{\textcolor{blue}{#1}}}

\newcommand{\quickwordcount}[1]{%
  \immediate\write18{texcount -1 -sum -merge -q #1.tex > #1-words.sum }%
  \input{#1-words.sum} words%
}

\newcommand{\quickcharcount}[1]{%
  \immediate\write18{texcount -1 -sum -merge -char -q #1.tex > #1-chars.sum }%
  \input{#1-chars.sum} characters (not including spaces)%
}

\newcommand{\detailtexcount}[1]{%
  \immediate\write18{texcount -merge -sum -q #1.tex > #1.wcdetail }%
  \verbatiminput{#1.wcdetail}%
}

%%%%%%%%%%%%%%%%%%%%%%%%%%%%%%%%%%%%%%%%%%%%%%%%%%

%%%%%%%%%%%%%%%%%%% TITLE PAGE %%%%%%%%%%%%%%%%%%%

% Title of the paper, and the short title which is used in the headers.
% Keep the title short and informative.
% \title[Short title, max. 45 characters]{MNRAS \LaTeXe\ Kinematics of PSB galaxies}
\title[Kinematics of PSB galaxies]{SDSS-IV MaNGA: Kinematics of PSB galaxies: detecting merger signatures} %[short title]{long title}

% The list of authors, and the short list which is used in the headers.
% If you need two or more lines of authors, add an extra line using \newauthor
\author[J. Proctor]{John Proctor$^{1,2}$\thanks{E-mail: jp210@st-andrews.ac.uk}
\\
% List of institutions
$^{1}$School of Physics and Astronomy, University of St Andrews, North Haugh, St Andrews KY16 9SS, UK\\
$^{2}$Royal Astronomical Society, Burlington House, Piccadilly, London W1J 0BQ, UK\\}

% These dates will be filled out by the publisher
\date{Accepted XXX. Received YYY; in original form ZZZ}

% Enter the current year, for the copyright statements etc.
\pubyear{2019}

% Don't change these lines
\begin{document}
\label{firstpage}
\pagerange{\pageref{firstpage}--\pageref{lastpage}}
\maketitle

% Abstract of the paper

\begin{abstract}

% Abstract of the paper

Post-starburst (PSB) galaxies can be identified from their spectra by the presence of strong Balmer series absorption lines and weak or absent emission lines. This is consistent with a population of A-type and older stars, indicating that star formation ceased within the past 1 to 2 Gyr. We analyse the kinematic properties of a sample of 68 PSB galaxies obtained from the Sloan Digital Sky Survey (SDSS) phase IV project: Mapping of Nearby Galaxies at Apache Point Observatory (MaNGA) integral field spectroscopic survey to determine if cessation of star formation in PSBs is consistent with evidence of major mergers. Major mergers may be revealed by asymmetries in the stellar and gas velocity fields. Firstly we look at differences in the kinematic position angles ($\Delta$PA$_{k}$) between the gas and stellar velocity fields for evidence of disruption. We then perform a Radon transform (RT) analysis to identify radial variation and asymmetry features present in the RT profile of PSB stellar velocity fields in an effort to identify underlying signatures of past merger activity.

The results of the kinematic position angle analysis show that PSBs exhibit a large range of $\Delta$PA$_{k}$ compared to a similar population of 'normal' or control galaxies. We conclude the $\Delta$PA$_{k}$ of PSBs show that they are a statistically different population from the control galaxies and this presents evidence of past disruption, possibly due to major mergers. The Radon transform analysis did not prove so conclusive however. Kinematic signatures of disruptive events consistent with major mergers are evident in a number of PSBs, however the data analysis technique will need to be refined in order to obtain more conclusive results. 

\end{abstract}

% Select between one and six entries from the list of approved keywords.
% Don't make up new ones.
\begin{keywords}
methods: data analysis, galaxies: kinematics and dynamics, galaxies: evolution
\end{keywords}

%%%%%%%%%%%%%%%%%%%%%%%%%%%%%%%%%%%%%%%%%%%%%%%%%%
% The MNRAS class isn't designed to include a table of contents, but for this document one is useful.
% I therefore have to do some kludging to make it work without masses of blank space.

%%TC:ignore
\begingroup
\let\clearpage\relax
\tableofcontents
% \listoftables
% \listoffigures
\endgroup
%%TC:endignore

% \newpage % [JP] remove the comment to remove the column break. 

%%%%%%%%%%%%%%%%% BODY OF PAPER %%%%%%%%%%%%%%%%%%

%%TC:ignore
\newpage
\section*{Word count}
\detailtexcount{main}
\newpage
%%TC:endignore

%%TC:ignore
\section*{Intro guidance}

This guidance text is not included in the word count and TODO: is to be removed.


Excellent start, thank you. Just to clarify the goal of this project is to use kinematic maps to ascertain if the post-starburst galaxies are caused by mergers (i.e. post-mergers). 

My apologies for the Swinbank reference. This one was the one I was thinking of: \citet{2012MNRAS.420..672S} One year out!
 
There are another couple of papers to look at:
Stark et al. 2018: \citep{2018MNRAS.480.2217S} 
Barrera-Ballesteros et al. 2015: \citep{2015A&A...582A..21B}.
 
Once you have looked at these, could you move up and down the references and cited papers in each paper, and see if you can find any other methods that have been used to identify merger or post-merger features \textbf{using kinematic features or maps}?
 
Extend your report to perhaps 1.5-2 pages to give a complete summary of the literature.
Remember that you are writing your report for a non-expert, so avoid jargon and explain symbols (i.e. K\_tot won't mean anything to the reader, or to me for that matter!). Whether a student can explain what they are doing to a non-expert is a key criteria for ascertaining whether they understood what they were doing, rather than just doing what their supervisor told them.  On discussion with Anne-Marie, we are not convinced that the full kinemetry fits will provide useful data on MaNGA galaxies. Note that it is important to get good marks on your final report that you provide a critical assessment of both your results and previous results, so have a think about the methods and what might work / not work on the MaNGA galaxies. 

\vspace{6pt}
\textbf{Remember to remove redundant subsection outlining placeholders.}

%%TC:endignore

\section{Introduction}
\label{sec:introduction}

\subsection{Galaxy evolution}
\label{sec:evolution}

The prevalent theory of galaxy evolution is often explained  by  by referring to galaxy colour-colour and colour-magnitude diagrams (CMD) \citep[see e.g.][]{2001AJ....122.1861S, 2003ApJ...585L...5H, 2003ApJS..149..289B,baldry2004quantifying,2006MNRAS.373..469B, }. A representative CMD constructed from SDSS observations of 20,000 nearby galaxies (z < 0.04)  is presented in  Figure~\ref{fig:CMD1}. The CMD exhibits a clear colour-magnitude bimodality.  Gas-rich, star-forming, often late-type galaxies (Sb, Sc, Irr) populate the so-called 'blue cloud' region of the CMD. As gas is consumed through star formation blue cloud galaxies are understood to transition to the redder mainly early-type (E, S0, Sa  quiescent galaxies along the upper left in the 'red sequence' region of the CMD. There is a sparsely populated region separating the blue cloud and red sequence populations often referred to as the 'green valley' region of the CMD \citep{2004ApJ...608..752B}. In this paper we explore the evolutionary pathway of galaxies in transition between the blue cloud and the red sequence, through the green valley.
As a visualisation of this scenario it is of interest to note that \citet{Mutch_2011} assert that large spiral Sa/SBa type galaxies like the Milky Way and the Andromeda galaxy M31 are in evolutionary transition from star-forming to passive galaxies due to the past consumption of much of their cold gas and now lie within the green valley. We will return to the CMD later after discussing the PSB sample selection criteria.

\begin{figure}
	\includegraphics[width=\columnwidth]{images/CMDs/galaxyCMD.PNG}
    \caption[Galaxy CMD]{Galaxy colour-magnitude diagram: the distribution of 20,000 nearby galaxies from the Sloan Digital Sky Survey. The colour index $u-g$ is plotted against $M_g$ magnitude. The distribution exhibits clear bimodality with late-type galaxies occupying a 'blue cloud' in the lower right while early-type galaxies form a 'red sequence' along the upper region of the diagram extending to brighter magnitudes. There is a clearly under-populated band separating the two dense regions, this is the so-called 'green valley'}
    \label{fig:CMD1}
\end{figure}


\begin{figure}
    \centering
    \includegraphics[width=\columnwidth]{images/CMDs/CMD-G_i-i.png}
    \caption{Caption}
    \label{fig:CMD-G_i-i}
\end{figure}

\begin{figure}
    \centering
    \includegraphics[width=\columnwidth]{images/CMDs/CMD-mass-1.png}
    \caption{Caption}
    \label{fig:CMD-mass-1}
\end{figure}

\subsection{Post starburst Galaxies}

Post-starburst (PSB) galaxies are a class of galaxies where the spectra indicate that star formation has ceased within the past gigayear or so. The is no spectral evidence of young O and B-type stars which will have gone supernova in the dynamical time since the starburst event. PSB spectra are dominated by the strong Balmer absorption lines H$\beta$, H$\gamma$, H$\delta$ indicative of a population of main sequence A- and F-type stars, particularly if a strong  H$\delta$ absorption line width is evident \citep{1997A&A...325.1025P}, while nebular emission lines characteristic of ongoing star formation, for example H$\alpha$ 6564 \AA\ and [OII] 3727\AA\, are weak or absent \citep{2001ApJ...547L..17B,2003PASJ...55..771G,2004MNRAS.355..713B,2005MNRAS.357..937G,2018MNRAS.477.1708P}. The morphology of PSBs is often that of early-type ellipticals and therefore PSBs are often referred to as E+A or K+A galaxies \citep{1983ApJ...270....7D,1996ApJ...466..104Z,2009ARA&A..47..159B}. These spectral observations are characteristic of a brief burst of star formation in the past 0.5 to 1.5 Gyr which has subsequently ceased, or been 'quenched' \citep{1983ApJ...270....7D,1987MNRAS.229..423C,1997A&A...325.1025P}. These spectral features are indicative that PSB galaxies are in transition on an evolutionary track from active star-forming galaxies to passive spheroids \citep{2004MNRAS.355..713B,2012MNRAS.420..672S,2013MNRAS.429.2212M}. Using a sample of a sample of low-redshift (z~0.1) E+A galaxies from the 2dF Galaxy Redshift Survey (2dFGRS) \citet{2004MNRAS.355..713B} find evidence of major mergers such as tidal tails and conclude that major mergers are an important formation mechanism for E+A PSB galaxies. A number of studies have shown that about half of galaxies have experienced recent rapid quenching while an evolutionary track to the red sequence \citep{Martin_2007,10.1111/j.1365-2966.2009.14537.x,2015MNRAS.450..435S}, however see \cite{2017ApJ...845..145W}. Quenching during mergers is also apparent in simulations, see e.g. \cite{2019MNRAS.484.2447D}.

In this study we analyse the kinematic properties of post-starburst galaxies to determine if there is evidence of major mergers in their kinematic signatures, and ask if major mergers could be the primary cause of the cessation of star formation. The objective would be to establish a link between merger activity, the quenching of star formation and morphological transition from blue late-type disc galaxies, through the green valley, to redder passive spheroids.

For this work we utilise a sample of 68 PSBs drawn from the SDSS-IV MaNGA (Mapping Nearby Galaxies at Apache Point Observatory integral field spectroscopic survey  as described in Section \ref{sec:data}.

[TODO: reformulate the following paragraph] See Vivienne's  papers researching PSBs: \citet{2017MNRAS.472.1401A} regarding the relationship of quenching and transition. \citet{2016MNRAS.463..832W} sets out the background work.

\subsection{Structure of the paper}
The content of the paper is organised as follows: A concise review of the literature on galaxy mergers and morphology transitions relevant to the paper is introduced in Section \ref{sec:mergers}. Details of the MaNGA survey and the selection criteria for our PSB sample and control galaxies are provided in Section \ref{sec:data}. Section \ref{sec:kinematics} describes the application of kinematics to the study of galaxy morphology and evolution. The selection criteria for our PSB sample and control galaxies are laid out in Section \ref{sec:data}. Data analysis methods and results are presented in Section \ref{sec:analysis}. The method of the velocity field analysis method employing the Radon transform method is a significant analysis tool in its own right. This is discussed in some detail in Section \ref{sec:Radon}. Finally, a summary of the research and the conclusions drawn from this work, along with recommendations for further study are presented in Section \ref{sec:discussion}.

\input{sections/data.tex}

\section{Mergers and morphology}
Here we present a review of the literature on the detection of mergers and post-mergers, using morphology alone.



\input{sections/m&m-guidance.tex}
\input{sections/kinematics.tex}

\subsection{kinemetry}
Kinemetry analysis employs the \texttt{kinemetry} software package developed by \citet{2006MNRAS.366..787K} to distinguish disc dominated systems from those exhibiting major mergers. The \texttt{Kinemetry} method involves mapping the gas velocity field and the gas velocity dispersion. Classification of a galaxy as a disc system or merger depends on the relationship between the gas velocity $v$ and the gas velocity dispersion $\sigma$.

\citet{2016A&A...591A..85B} examine the gas kinematics of nearby (ultra)luminous infrared galaxies ((U)LIRGs) at $z<0.1$. The objective is to analyse the kinematic properties of local (U)LIRGs to characterise their structures and thereby classify those (U)LIRGs as having disc structures (disc class), or displaying evidence of major merger activity (merger class). Their method employs optical integral field spectroscopy (IFS) data obtained at the VLT. H$\alpha$ emission is used as a gas velocity tracer. \citet{2016A&A...591A..85B} conclude that their results confirm that well-defined discs can be effectively distinguished from well-defined mergers but there is intermediate, indeterminate class. They note that the \texttt{kinemetry} method is sensitive to angular resolution of the integral field unit (IFU). \citet{2008ApJ...682..231S} had earlier performed an  analysis of warm gas kinematics as traced by H$\alpha$ emission, but concentrated on sample at $z\sim2$ using the NIR IFS instrument SIMFONI on the VLT. 


%% \section{Methods}
\label{methods}

\subsection{Kinemetry}





\subsection{Radon transform}

 Empty outline..

\section{The Radon transform}
Describe the principles of the Radon transform, its origins and history e.g. \citet{radon1917determination} and  \citet{7910dc8d5b654c90ac4bc94c67d06f01}, and its purpose and relevance to this research. Refer to \cite{2018MNRAS.480.2217S}.

\begin{figure}
    \centering
    \includegraphics[width=\columnwidth]{images/RadonPlots/Radon-transform-Stark.png}
    \caption{The Radon transform: the figure illustrates the Radon transform and its coordinate system as described in \citet{2018MNRAS.480.2217S}. Line integrals across an image are calculated along all possible lines, parameterised by the coordinates [$\theta$, $\rho$], that cross the 2D function v(x, y). Two examples are shown in the left-hand panel, where the integrals are calculated over the solid lines, L$_1$ and L$_2$, which are perpendicular to the [$\theta, \rho$] vectors and mapped to the points $\theta_1$, $\rho_1$ and $\theta_2$, $\rho_2$ in [$\theta$, $\rho$] parameter space in the right-hand panel. In the Radon transform coordinate system, $\theta$ ranges from 0 to 180 $\deg$ while $\rho$ ranges from $-\infty$ to $\infty$  such that a position below the x-axis corresponds to $\rho < 0$.}
    \label{fig:RadonTransform}
\end{figure}

\citet{2018MNRAS.480.2217S} have applied the Radon transform technique to stellar and gas velocity fields obtained from the MaNGA integral field survey in order to quantify radial variations in the kinematic position angles of galaxies (PA$_k$) using the Radon transform method \citep[see e.g.][]{radon1917determination, 7910dc8d5b654c90ac4bc94c67d06f01}. 
The Radon transform is defined as

\begin{equation}
    \label{eqn:radon}
    R(\rho,\theta)=\int_{L}{v(x,y)\, \diff l}.
\end{equation}

This transform is illustrated graphically in Fig. \ref{fig:RadonTransform}. Lines through an image in 2D velocity space are mapped to a series of points in Radon transform [$\theta,\rho$] space by means of line integrals at various angles about the velocity space axes, and at various distances offset from the velocity space origin. [TODO: expand on the definition of the RT here...]. 

An example of the graphical output of the \citet{2018MNRAS.480.2217S} Radon transform code \texttt{ds\_radon.pro} as applied to a synthetic velocity field is shown in Fig. \ref{fig:Radon}.

\begin{figure}
    \centering
   	\includegraphics[width=\columnwidth]{images/RadonPlots/example2.png}
    \caption{Model Radon transform plots as described in \citet{2018MNRAS.480.2217S}. The left panel shows a synthetic uniform velocity field model, the middle panel shows the the absolute Radon transform of the velocity field and the right panel shows the aperture-restricted absolute transform. We are concerned with the latter, the absolute aperture-restricted  Radon transform of stellar and gas velocity fields in this work.}
    \label{fig:Radon}
\end{figure}


\begin{figure}
    \centering
   	\includegraphics[width=\columnwidth]{images/RadonPlots/RT-snips/CPSB-8313-6101-RT-snip.png}
    \caption{Radon transform plots of the stellar velocity (left panel) and gas velocity (right) fields of CPSB of MaNGA PLATEIFU 8131-6101}
    \label{fig:RT_8131-6101}
\end{figure}



\begin{figure}
    \centering
    \includegraphics[width=\columnwidth]{images/RadonPlots/RT-snips/CPSB-8313-6101-RT-snip.png}
    \includegraphics[width=\columnwidth]{images/RadonPlots/RT-snips/CPSB-9494-3701-snip.png}
    \includegraphics[width=\columnwidth]{images/RadonPlots/RT-snips/CPSB-8398-6102-snip.png}
    \caption{CPSBs: Radon transforms of stellar velocity and gas velocity maps. From the top CPSB-8313-6101, CPSB-9404-3710 and CPSB -8398-6103}
    \label{fig:CPSB-RTs}
\end{figure}
\section{Results and analysis}
\label{sec:analysis}

\subsection{Property distributions}
Here we will look at the frequency distributions (histograms) comparing the characteristics of CPSBs, RPSBs an their control galaxies.

The surface brightness profile of a galaxy is generally a function of radius from the nucleus. The rate of change of brightness can be described by the S\'ersic function or S\'ersic  index (SI). Low values of SI $\sim$1 relate to disc-type galaxies, whereas SI values of 3 or more are measured in galaxies with compact bright nuclei, indicative of more evolved systems.  

We compare the distributions of CPSBs with that of the RPSB sample in terms of S\'ersic index, stellar mass $M_*$, and redshift z. These distributions are taken from the data in Tables \ref{tab:my-CPSBs} and \ref{tab:my-RPSBs}. The frequency distribution histograms are plotted in Figures \ref{fig:Sersic-plot}, \ref{fig:stellar-mass-plot} and \ref{fig:redshift-plot} respectively.

The distribution of S\'ersic index values as shown in Figure \ref{fig:Sersic-plot} covers a wide range of values indicating a wide range of morphologies. CPSBs  generally have higher S\'ersic index values than RPBs. This suggests that CPBs tend to have concentrated nuclei and a spheroidal morphology contrary to the more disc-like RPSBs with lower SI-n values.

\begin{figure}
    \centering
    \includegraphics[width=\columnwidth]{images/JupyterPlots/Dist-Sersic-Index-All.png}
    \caption{Distribution of S\'ersic index (SI) values. The relative number density distribution of the S\'ersic index value for 26 CPSBs (red histogram) and 36 RPSBs (blue) are plotted. Five PSBs with unreliable SI values (SI = 6.000) in the NSA data  have been excluded.}
    \label{fig:Sersic-plot}
\end{figure}

\begin{figure}
    \centering
    \includegraphics[width=\columnwidth]{images/JupyterPlots/Dist-Stellar-Mass-All.png}
    \caption{Distribution of Stellar mass of our sample of CPSBs (rad) and RPSBs (blue). On the scale of $\log_{10}$\Msun\ a fairly uniform distribution of stellar mass is apparent for both populations, CPSBs and RPSBs.}
    \label{fig:stellar-mass-plot}
\end{figure}

\begin{figure}
    \centering
    \includegraphics[width=\columnwidth]{images/JupyterPlots/Dist-z-All.png}
    \caption[PSB distribution in redshift]{Distribution in redshift: the redshift z as obtained from the NSA.Z data value for the CPSB sample (shaded in red) is compared to the RPSB sample (blue).}
    \label{fig:redshift-plot}
\end{figure}

\begin{figure}
    \centering
    \includegraphics[width=\columnwidth]{images/JupyterPlots/Dist-Delta-PA-All-GoodFlags.png}
    \caption[Distribution of PSB velocity field position angles]{Distribution of PSB galaxy velocity map position angles for those PSBs with stellar velocity and gas velocity characteristics flagged as 'good' as denoted in the legend (details are provided in the text). CPSB PA density weights are plotted in red, RPSB PA weights in blue.}
    \label{fig:deltaPAdistribution}
\end{figure}

The velocity field position angle variance of the CPSB and PSB control galaxies is shown in Figure \ref{fig:controlDeltaPAs}. Both sets of control galaxies have low values of velocity field $\Delta$PAs, i.e. the stellar velocity and gas velocity fields are generally aligned.

\begin{figure}
    \centering
    \includegraphics[width=\columnwidth]{images/JupyterPlots/Distribution-of-control-galaxy-deltaPA.png}
    \caption[Distribution of control galaxy $\Delta$PAs]{Distribution of control galaxy stellar and gas velocity field $\Delta$PAs.}
    \label{fig:controlDeltaPAs}
\end{figure}

\begin{figure}
    \centering
    \includegraphics[width=\columnwidth]{images/JupyterPlots/Distribution-of-CPSB-dPA-vs-controls.png}
    \caption{Distribution of CPSB stellar-gas velocity $\Delta$PA vs. controls.}
    \label{fig:CPSBvsControlDeltaPAs}
\end{figure}

\begin{figure}
    \centering
    \includegraphics[width=\columnwidth]{images/JupyterPlots/Distribution-of-RPSB-dPA-vs-controls.png}
    \caption{Distribution of RPSB stellar-gas velocity $\Delta$PA vs. controls.}
    \label{fig:RPSBvsControlDeltaPAs}
\end{figure}

\subsection{The KS-test}
The Kolmogorov-Smirnov test: a statistical test to determine if two samples come from the same underlying distribution, see e.g. \citet{hodges1958significance}. We implement this test using the SciPy package \texttt{scipy.stats.kstest} module\footnote{\href{}{https://docs.scipy.org/doc/scipy/reference/generated/scipy.stats.ks\_2samp.html}}.

We ran the two-sided K-S test statistic on the distribution of $\Delta$ position angles obtained from the \texttt{kinemetry} analysis. The results are given in Table \ref{tab:K-S-tests}. The statistical significance inferred from the from the K-S test is that a high value of the test statistic, typically > 0.1, together with a low p-value, < 0.1 indicates that the samples are drawn from different statistical distributions. Here we note that the CPSB and RPSB samples originate from different distributions, and also each sample is from a different distribution from its corresponding control galaxy sample. The Python implementation \texttt{scipy.stats.ks\_2samp} accepts samples of different sizes.

\begin{table}
\caption[Kolmogorov-Smirnov statistical test]{Kolmogorov-Smirnov statistical test on various $\Delta$PA sample distributions. A high value of the K-S statistic > 10\%, together with a low p-value, < 10\% indicates that the samples come from different statistical distributions.}
\label{tab:K-S-tests}
\begin{tabular}{llcc}
\hline
$\Delta$PA sample 1  & $\Delta$PA sample 2 & K-S statistic & p-value \\
\hline
CPSB & RPSB & 0.467 & 0.040 \\
CPSB & CPSB controls & 0.755 & 0.000 \\
RPSB & RPSB controls & 0.520 & 0.000 \\
CPSB controls & RPSB controls & 0.197 & 0.001 \\
\hline
\end{tabular}
\end{table}

\subsection{Radon profile classification}
\label{sec:Radon-profile-classification}

A summary of the final visual classification method is shown in Table \ref{tab:Radon-class-summary}. The results of the Radon profile classification process for each of the individual target galaxies is provided in Appendix \ref{sec:visual-classification-tables}. 

\begin{table}
    \centering
    \caption{Summary of the classification of Radon profile types as categorised visually in the CPSB and RPSB groups and their control samples. A few galaxies could not be visually classified (NC) through the analysis due to masked areas in the MaNGA maps.}
    \label{tab:Radon-class-summary}
    \begin{tabular}{lc}
    \hline
    Radon profile type & Number in classification \\
    \hline
    Type-A: Asymmetric & 25 \\
    Type-C: Constant & 38 \\
    Type-IB: Inner Bend & 22 \\
    Type-OB: Outer Bend & 17 \\
    Type-OB+IB: Outer and Inner Bends & 17 \\
    NC: Not classified & 8 \\
    \hline
    \end{tabular}
\end{table}

On completion of the Radon profile classification we investigate the distribution of the Radon profile types in each of the PSB categories (CPSB and RPSB) and their control samples (CPSB-controls and RPSB-controls). As mentioned earlier the visual classification was performed on the merits of the Radon output plots, Radon Transform (RT) and Radon profile trace without prior knowledge of the PSB group classification. A summary of the Radon profile visual classification assessments grouped by galaxy group category is provided in Table \ref{tab:Radon-VC-results}.

\begin{table*}
\caption{Results of the Radon profile visual classification assessments for the CPSB and RPSB samples and their control galaxies.  PSBs and their controls were assigned a Radon profile according to the appearance of the shape of the Radon profile trace plot. Where clear features were evident the galaxy was assigned a Radon profile Type as described in the text. Galaxies that were not classified due to poor data were flagged marked as NC. The percentage of each Type of those classified in the group shown in parentheses.}
\label{tab:Radon-VC-results}
\begin{tabular}{lccccccc}
\hline
 & \begin{tabular}[c]{@{}c@{}}Classified \end{tabular} & Constant & Inner Bend & Outer Bend & \begin{tabular}[c]{@{}c@{}}Inner Bend + \\ Outer Bend\end{tabular} & Asymmetric & \begin{tabular}[c]{@{}c@{}}Not\\ Classified\end{tabular} \\
Galaxy group &  & (Type-C) & (Type-IB) & (Type-OB) & (Type-IB+OB) & (Type-A) & (NC) \\
 \hline
CPSBs & 27 & 7 (26\%) & 6 (22\%) & 5 (19\%) & 5 (19\%) & 4 (15\%) & 1 \\
CPSB controls & 29 & 6 (21\%) & 7 (21\%) & 6 (21\%) & 5 (17\%) & 5 (17\%) & 2 \\
RPSBs & 36 & 14 (39\%) & 5 (14\%) & 5 (14\%) & 5 (14\%) & 7 (19\%) & - \\
RPSB controls & 31 & 10 (32\%) & 4 (13\%) & 3 (10\%) & 4 (13\%) & 10 (32\%) & 5 \\
\hline
Totals & 123 & \multicolumn{1}{l}{37 (30\%)} & 22 (18\%) & 19 (15\%) & 19 (15\%) & 26 (21\%) & 8 \\
\hline
\end{tabular}
\end{table*}

[TODO: analyse the results in Table \ref{tab:Radon-VC-results}]

The Radon profile classification results summarised in Table \ref{tab:Radon-VC-results}  show that galaxies with bend features, Type-IB, Type-OB and Type-IB+OB, are clearly more prevalent in CPSBs (16 out of 27, or 59\% of galaxies classified) than in RPSBs (15 out of 36, or 42\% of those classified). A similar trend is found in the control groups: CPSB controls exhibit 18 out of 29, or 62\% with bend features, while RPSB controls have 11 out of 31, or 35\% with bend features.

Stark et al. (2018) [TODO: citation] conclude that Radon trace profiles may be linked to kinematic and morphological features in the following manner. 



\section{Discussion and conclusions}
\label{sec:discussion}
We have four topics to discuss here:
\begin{itemize}
\item \textbf{Velocity map binning methods:} a technical discussion on the comparative merits of alternative MaNGA velocity map binning methods and IFU sizes to obtain best resolution Radon profile traces.
\item \textbf{Correlation between Radon profiles classes and morphology:} we discuss a possible relationship between Radon transform trace profile Types and galaxy morphology.
\item \textbf{Findings and conclusions:} a review of the significant findings of the project and presentation of our conclusions.
\item \textbf{Future work:} some topics for future work noted during the course of the project. 
\end{itemize}

\subsection{Velocity map binning methods}
\label{sec:binning-methods}
The MaNGA DAP utilises various spaxel binning models to process the IFU fibre bundle data into the output maps. The available binning schemes used to produce gas and stellar velocity maps, together with a brief description of each are listed below.

\begin{itemize}
    \item SPX - single spaxel measurements i.e. no binning.
    \item VOR10 - Voronoi binning, an adaptive spatial binning method where low signal-to-noise (S/N) ratio spaxels are grouped to achieve an overall S/N of 10  \citep{2003MNRAS.342..345C, 2019arXiv190100856W}.
    \item HYB10 - Hybrid binning: Voronoi S/N 10 binning for stellar velocity maps, but unbinned for emission line properties which are used to generate gas velocity maps.
\end{itemize}  

In this study we have utilised MaNGA datacubes released in DR15 MPL-7 which have been processed in the DAP using the hybrid HYB10 binning model, i.e. using VOR10 binning for the stellar velocity maps. However, the MaNGA project team currently have an internal dataset available, MPL-8. Datacubes in MPL-8 include VOR10 and SPX binning schema. We were interested to make a comparison of the Radon trace profiles generated from the datacubes using the Voronoi and SPX binning schemas. A limited sample of MPL-8 cubes were made available for this comparison exercise. We selected 2 galaxies from our samples with stellar velocity maps having poor spatial definition, and another 2 with good resolution in order to compare the Radon transform profiles using the Voronoi binning and SPX (unbinned) methods. The selected datacubes were firstly, examples of good resolution/definition in stellar velocity maps:

\begin{itemize}
    \item 7977-12704
    \item 8322-1901
\end{itemize}

and secondly, examples of poor resolution/definition in stellar velocity maps, i.e. having a blotchy appearance, due to spatial binning of low S/N spaxels, or with masked spaxels:

\begin{itemize}
    \item 9088-12703
    \item 8993-6104
\end{itemize}

The stellar velocity maps together with their Radon transform and trace profiles for these well defined example 7977-12704 are shown in Figure \ref{fig:binning-comparison}. The differences being the binning models of the input stellar velocity maps: Voronoi binning in the left-hand panel and SPX binning on the right. A comparison shows that there is little difference in the Radon transform and Radon profile plots. It is also apparent that the Radon transform algorithm for the SPX velocity maps yields a greater number of valid data points in the trace plot. We conclude that both VOR10 and SPX binned maps would lead to the same Type classification in these cases. This conclusion was also obtained for other galaxy showing good resolution in the stellar velocity map,  8322-1901 (figure not included). 

However, the same conclusion does not apply for the low resolution stellar velocity map examples, 8993-6104 shown in Figure \ref{fig:binning-comparison2} and 9088-12703 (figure not included). In these cases, at first sight, there are marked differences in the extent of spatial coverage and shapes of the velocity field PA, the Radon transform minimum and the Radon profile. In particular the control galaxy 8993-6104 displays a clearly asymmetric Radon profile with VOR10 binning, but could possibly be interpreted as an outer bend in the Radon profile trace of the SPX map. However, on closer inspection, looking at the  range the trace plots, the same bend feature at Radon transform coordinates [$\rho$, $\theta$]=[-6,+2],[20, 50] is apparent in both binning model plots. The asymmetric feature at $\rho$ \textgreater\ +2 apparent in the left Voronoi binned trace is not evident in the SPX output shown in the right-hand panel. Although these differences can be expected due to the exclusion of low S/N low spaxels in the SPX maps, crucially this can easily lead to a different visual classification result between the transforms of Voronoi binned and SPX velocity field maps. 
\begin{figure*}
    \centering
    \includegraphics[width=\columnwidth]{images/RadonPlots/RT-SNIPS-NEW/7977-12704-VOR10-MILESHC-MILESHC-1-SNIP.png}
    \includegraphics[width=\columnwidth]{images/RadonPlots/RT-SNIPS-NEW/7977-12704-SPX-MILESHC-MILESHC-1-SNIP.png}
    \caption[Comparison of velocity map binning schemes for a high resolution map]{Comparison of stellar velocity map binning methods using the Radon transform output graphic for the galaxy 7977-12704. The velocity map has good resolution using both schemes. In the left panel the transform RT and profile trace plots are generated from the stellar velocity map with Voronoi binning. The plots in the right panel were produced from the stellar velocity map using single spaxel SPX resolution. The layout of the 4 subplots in each panel is as described in Figure \ref{fig:8442-3704-complete}.}
    \label{fig:binning-comparison}
\end{figure*}

\begin{figure*}
    \centering
    \includegraphics[width=\columnwidth]{images/RadonPlots/RT-SNIPS-NEW/8993-6104-VOR10-MILESHC-MILESHC-1-SNIP.png}
    \includegraphics[width=\columnwidth]{images/RadonPlots/RT-SNIPS-NEW/8993-6104-SPX-MILESHC-MILESHC-1-SNIP.png}
    \caption[Comparison of velocity map binning schemes for a low resolution map]{Comparison of velocity map binning schemes for a low resolution map. Radon transform output graphic for the galaxy 8993-6104 with a low resolution stellar velocity map. In the left panel the transform RT and profile trace plots are generated from the stellar velocity map with Voronoi binning. The plots in the right panel were produced from the stellar velocity map using single spaxel SPX binning. The layout of the 4 subplots in each panel is as described in Figure \ref{fig:8442-3704-complete}.}
    \label{fig:binning-comparison2}
\end{figure*}




\subsection[Correlation between Radon profiles and morphology]{Correlation between Radon profile types and morphological features}
\label{sec:correlations}
\cite{2018MNRAS.480.2217S} made considerable efforts to explore  possible correlations between the observed frequencies of the 5 Radon profile types (Type-C, Type-A, Type-IB, Type-OB and Type-OB+IB) and underlying morphological features obtained from the Galaxy Zoo morphology classification project. They found some evidence of associations between Radon profile trace type and morphological features, as discussed in detail in their paper, Section 5. These correlations are subjective and certainly not exclusive of other possible Radon type to morphology relationships. Care must be taken to follow their arguments closely. The most prevalent of the identified relationships can be broadly interpreted as follows:
\begin{itemize}
\item Type-C: Constant profile - often apparent in unbarred galaxies.
\item Type-A: Asymmetric profile - can be associated with tidal interactions.
\item Type-IB: Inner bends - prevalent in strongly barred galaxies.
\item Type-OB: Outer bends - associated with kinematic warps.
\end{itemize}

Following the above interpretation there are two relevant points here. Firstly that Radon profile Type-OB outer bend features have an association with kinematic warps, and in this way could represent the signature of past mergers in disc galaxies. Secondly, following the same argument, Radon Type-A asymmetric profiles can be associated with tidal interactions. Tidal tails are signatures of gas-poor 'dry' mergers in early-type galaxies, two thirds of which are remnants \citep{2005AJ....130.2647V}. The post-merger remnants of galaxies with Type-OB or Type-A Radon profiles may be the candidates for our quenched post-starburst systems.

As reported in our results Section \ref{sec:comparison-of-results} we found that 5 out of 14 PSBs with large $\Delta$PA$_{k}$ offsets were assigned Radon profile trace classification of Type-OB. We emphasise that this result was determined by Classifier A, and there was little consensus with the classifications of either Classifier B, or in the limited number of automatic classifications that were available for our galaxies. Setting this point aside, some correlation may exist between Radon outer bend profiles, large $\Delta$PA$_{k}$ and past major mergers. Again we stress that much further work is required to provide consistent interpretations of Radon profile trace types before we can make this conclusion.

\subsection{Findings and conclusions}
\label{findings}

We have shown that CPSBs generally show a wide range of kinematic position angle differences $\Delta$PA$_{k}$ \textgreater 30\textdegree\ with many in the range 90 \textless\ $\Delta$PA$_{k}$ \textless\ 180\textdegree. RPSBs show a smaller range in $\Delta$PA$_{k}$ with only a few over 90\textdegree. The majority of control galaxies have small kinematic position angle differences, clustered below 25\textdegree. Significant misalignment in the gas and stellar velocity fields of CPSBs, and to a lesser extent in RPSBs, reveals kinematic disturbances indicative of past major mergers.
We noted that the distribution $\Delta$PA$_{k}$ is markedly different comparing CPSBs with CPSB controls, and trait is present but to a lesser degree in the comparison of the RPSB $\Delta$PA$_{k}$ with their controls. The distribution in differential kinematic position angles is similar for both control sets. These observations are strongly supported by the results of K-S analysis in Section \ref{sec:K-S-test} where we found the difference in $\Delta$PA$_{k}$ of both PSB groups as compared to their control groups to be statistically significant. 

In summary, our most significant finding from the kinematic position analysis is that there is a combination of mutually supportive evidence derived from the properties of PSBs that mergers play a role in galaxy evolution. This evidence was found in three topics: the $\Delta$PA$_{k}$ distributions in the CPSB, RPSB and their control sets; the supporting K-S statistical analysis performed on these $\Delta$PA$_{k}$ distributions; and the differences in distributions of S\'ersic index. Taken together these indicators reveal an evolutionary sequence from normal, non-PSB (control) galaxies, to the lower S\'ersic index disc-like RPSBs, and on to the redder spheroidal CPSBs with higher S\'ersic indices. We see PSBs as the indicators of morphological transition from late-type disc galaxies to red spheroids, and demonstrating an evolutionary pathway from the blue cloud, through the green valley, and on towards the red sequence.

In a mainly technical discussion on the relative merits of MaNGA map binning schema we compared the Radon profile trace results obtained from stellar velocity maps employing Voronoi VOR10 binning versus the single spaxel SPX unbinned schema. We obtained cleaner trace profiles (i.e. easier to classify) with a greater number of valid data points and tighter error bars from high S/N maps based on single spaxel SPX binning. For future studies, given a large enough sample of PSBs, we should avoid Voronoi binned maps where possible and use SPX maps a cut of S/N \textgreater say 3-5 in single spaxels.

Comparing the methods of kinematic PA analysis versus Radon transform, as mentioned in Section \ref{sec:motivation}, we noted that the Radon transform method provides distinct advantages over the kinematic $\Delta$PA$_{k}$ method. Radon transforms provide a detailed trace of position angle across the velocity field enabling local regions of radial variation to be identified. Kinematic PAs require both stellar and gas velocity fields to be mapped but post-starburst galaxies may possess little gas. The Radon transform can identify local regions of disturbance, as evidence of merger activity, using the stellar velocity field alone.

In an earlier part of this discussion (Section  \ref{sec:correlations}) we put forward the notion that Radon profile Type-OB features may be consistent with kinematic warps in stellar discs, while Type-A profiles suggest the presence of tidal tails from interacting or merged early-type field galaxies. In both cases merged systems of sufficient age will reveal the characteristics of post-starburst galaxies. We draw a loose inference here that Radon trace profiles displaying outer bends, Type-OB or Type-OB+IB (suggesting warped stellar discs) could be considered prospective candidates of past major mergers in disc-dominated systems, while Type-A profiles (consistent with tidal interactions) may correlate with ongoing or past mergers in elliptical systems. From Table \ref{tab:Radon-VC-results} we see that Classifier A identified 10 of of 27 CPSBs (37\%), and 10 out of 36 RPSBs (28\%) as having Type-OB or Type-OB+IB Radon profile signatures. Type-A features were evident in  15\% of CPSBs and 19\% of RPSBs. If this classification is reliable and the correlation between type and morphology is valid, very big if's however, then we conclude that there is some evidence of past mergers in post-starburst galaxies. This, of course, is subject further research into the areas of considerable uncertainty: Radon classification and the correlation relationship between Radon type and morphology.

Visual classification of Radon profile types was found to be very difficult, our classifiers agreed on this point. In addition our classifiers arrived at varying interpretations of profile type class for most of the galaxies. Although the same written classification procedure was followed, the method was quite loosely defined, recognising the difficulties in relating the race plots to the examples. With so few classifiers involved we conclude that there is significant classification error present in our results, remembering the the Galaxy Zoo project engaged $\sim10^5$ classifiers to minimise classification error.

The results of the visual classification of the 5 Radon trace type profiles for CPSBs, RPSBs and their control groups as presented in Table \ref{tab:Radon-VC-results} and shown graphically in Figure \ref{fig:Radon-grouped-barchart} are quantitatively similar in number. This was unexpected and is presently not fully explained. We can, however, expect more galaxies with constant features in the RPSB groups than the CPSBs as RPSBs exhibit PSB features in local regions only while we can expect a higher percentage of the non-linear features, i.e. inner and outer bends and composite bend features Type-OB+IB, to be apparent in the trace profiles of CPSBs due to their central and more widespread post-starburst regions. 

% \subsection{Summary}
% \label{summary}
In previous studies major mergers have been detected using imaging techniques. In this project we have attempted to identify past major mergers in PSB galaxies, firstly by investigating the distributions of differences in stellar and gas velocity field kinematic position angles, and secondly by using the Radon transform method to reveal radial variation in kinematic position angles. This analysis did not prove conclusive. More work is required to provide consistent classification of kinematic features apparent in the Radon profile trace plots. It is noted however that \cite{2019DDA....5020304N} have recently announced work which promises to increase the accuracy of merger detection using a method that integrates imaging and kinematic analysis techniques. This extended method will combine SDSS imaging (providing morphological observations) with MaNGA  kinematic maps, towards enhanced identification of merger and post-merger signatures. When available, this technique should be applied to our PSB and control samples to provide additional evidence of past mergers in post-starburst galaxies.


\section*{Acknowledgements}

The Acknowledgements section is not numbered. Here you can thank helpful
colleagues, acknowledge funding agencies, telescopes and facilities used etc.
Try to keep it short.

%%%%%%%%%%%%%%%%%%%%%%%%%%%%%%%%%%%%%%%%%%%%%%%%%%

%%%%%%%%%%%%%%%%%%%% REFERENCES %%%%%%%%%%%%%%%%%%

% The best way to enter references is to use BibTeX:

%\bibliographystyle{mnras}
%\bibliography{example} % if your bibtex file is called example.bib
%----------- REFERENCES ------------------%
\bibliographystyle{mnras}
\bibliography{JPbib2019} 


%%%%%%%%%%%%%%%%%%%%%%%%%%%%%%%%%%%%%%%%%%%%%%%%%%

%%%%%%%%%%%%%%%%% APPENDICES %%%%%%%%%%%%%%%%%%%%%

% include your \appendix here

% \appendix
% \section{Visual classification of Radon profiles}
\label{sec:visual-classification-tables}

%TC:ignore

Include a table here with the results of the Radon profile visual assessment as described in Section \ref{sec:Radon-classification} 

\begin{table}[]
    \centering
    \begin{tabular}{c|c}
         &  \\
         & 
    \end{tabular}
    \caption{Caption: The table reflecting the contents of the classification spreadsheet.}
    \label{tab:visual-class-table}
\end{table}

%TC:endignore


%%%%%%%%%%%%%%%%%%%%%%%%%%%%%%%%%%%%%%%%%%%%%%%%%%


% Don't change these lines
\bsp	% typesetting comment
\label{lastpage}
\end{document}

% End of mnras_template.tex