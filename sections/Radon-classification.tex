\subsection{Radon profile classification procedure}
\label{sec:Radon-classification}

\cite{2018MNRAS.480.2217S} identified 5 commonly recurring patterns in the stellar and gas velocity field Radon transform profiles of their MaNGA galaxy sample. These patterns were used to classify the computed Radon trace profiles. In this project work we adopt the same classification approach for the Radon profiles of our PSB galaxies and their control samples. Simplified models of 4 of the trace profile classes are shown in Figure \ref{fig:class-models}. In addition an asymmetric profile class was identified. Detailed examples of the trace profile class types are presented in Figure 7 of \cite{2018MNRAS.480.2217S}. The salient features of these 5 Radon profile classes are listed below:

\begin{itemize}
    \item Constant, \textbf{Type-C} : Radon profile with relatively constant trace minimum angle $\hat{\theta}$ at all radii $\rho$.
    \item Inner Bend, \textbf{Type-IB} : Galaxies whose Radon profiles exhibit symmetrical variations of $\hat{\theta}$ beginning at $|\rho|=0$, then transitioning to a constant value. 
    \item Outer Bend, \textbf{Type-OB} : Galaxies with constant Radon trace angle $\hat{\theta}$  at small $|\rho|$ which transition to a different value at a greater radius. 
    \item Inner Bend + Outer Bend, \textbf{Type-IB+OB} : Galaxies with Radon profiles showing a combination of the features of Type-IB and Type OB profiles.
    \item Asymmetric, \textbf{Type-A} : The value of the $\hat{\theta}$ varies significantly with $\rho$ across opposite sides of the transform R\textsubscript{AB}. 
 \end{itemize}

\begin{figure}
    \centering
    \includegraphics[width=0.8\columnwidth]{images/RadonPlots/Radon-class-models.png}
    \caption[Radon profile class feature identification: toy models]{Toy models of the Radon profile (trace angle minimum $\hat\theta$ versus radius $\rho$). The sub-plots display examples of 4 of the Radon profile classes used for classification of the RT trace plots. Upper left: Constant, Type-C; upper right: Inner Bend, Type-IB; lower left: Outer Bend, Type-OB; and lower right: Inner Bend + Outer Bend, Type-IB+OB. Source: Stark et al. (2018) figure 8.}
    \label{fig:class-models}
\end{figure}

Mathematical functions describing these Radon profile classifications have been identified by \citet[][section 3.6]{2018MNRAS.480.2217S}. This has enabled code routines to be developed which can provide automatic classification of the Radon trace profiles for galaxies in MaNGA Product Launch MPL-5. The results of the automatic classification routines were made available later in the progress of this project work and therefore have not been used in this analysis. Instead, we have adopted a simple visual classification method to categorise our sample galaxy into one of the 5 Radon profile trace types listed above. The visual classification process is described in the following 3-step process:

\begin{enumerate}
    \item Firstly we obtain the MAPS data cube FITS files for the selected PSB galaxies listed in Tables \ref{tab:my-CPSBs} and \ref{tab:my-RPSBs}, and a similar number of 'normal' galaxies drawn from their respective control samples, as described in Section \ref{sec:controls}, downloaded from the MaNGA website.
    \item Next, we process each of data cubes through the Radon transform wrapper code to obtain graphical output files showing the galaxy SDSS $gri$ image cutout, the MaNGA stellar velocity map, the absolute bounded Radon transform R\textsubscript{AB} plot and the Radon profile plot of $\hat{\theta}$ versus $\rho$. An example of this output for the CPSB galaxy 8979-1902 is shown in Figure \ref{fig:CPSB-8979-1902-SNIP}. 
    \item  Finally, we examine the output plot for each galaxy and visually assess the relative qualitative strength of each of the 5 classification features by assigning a numeric weighting as given in Table \ref{tab:features}. This method adds a semi-quantitative approach to the visual assessment process.
\end{enumerate}

\begin{figure*}
    \centering
    \includegraphics[width=0.8\textwidth]{images/RadonPlots/RT-SNIPS-NEW/CPSB-8979-1902-SNIP.png}
    \caption[Radon transform code output graphics for the CPSB 8979-1902]{Radon transform code output graphics for the CPSB 8979-1902. Top left: SDSS gri image cutout, Top right: stellar velocity map with kinematic position angle (PA$_{k}$) shown as the magenta line, lower left: Radon transform (RT) of the stellar velocity field with the transform minimum angle $\hat\theta$ plotted across radius $\rho$ in magenta, lower right: the Radon profile trace of the RT minimum.}
    \label{fig:CPSB-8979-1902-SNIP}
\end{figure*}

\begin{table}
    \caption[Relative weighting of Radon profile feature strengths used in visual classification]{Relative weighting of Radon profile feature strengths used for the visual classification of Radon profile feature types: Constant, Inner Bend, Outer Bend, Inner Bend + Outer Bend or Asymmetric. The weighting is assigned to help quantify the visual appearance of the trace profile plots.}
    \label{tab:features}
    \centering
    \begin{tabular}{cl}
    \hline
    Value & Visual appearance \\
    \hline
    2 & The feature is visually predominant \\
    1 & Some evidence of the feature is apparent \\
    0 & The feature is absent \\
    \hline
    \end{tabular}
\end{table}

The Radon output graphic plots for all 127 the galaxies were visually assessed in alphanumerical order of their PLATEIFU file name tag, which effectively creates a random sequence. This is intended to avoid introducing a selection bias by assessing all galaxies in each group separately. one complete group at a time, CPSBs, RPSBs, or their respective control sample groups, where commonly recurring features may be apparent in a particular group but not in the others.

After inspecting the Radon transform and associated Radon trace plots for a particular galaxy an assessment of the visual strength of the Radon profile type features evident in the plots. A numerical value 0, 1 or 2, representing the visual strength of apparent features from Table \ref{tab:features}, is assigned to each of the 5 predefined Type classes for the galaxy. Based on the relative strength values allocated a predominant feature Type (C, IB, OB, IB+OB or A) is assigned to that particular galaxy. To determine the relative predominance of Type-IB+OB features we simply sum the strength values assigned to Types IB and OB together. 
In many cases there is some uncertainty in the absolute Type assessment, i.e. only one of the 5 defines classes, a secondary assessment is made,  generally this is the primary Type plus a less evident Type, or sub-dominant Type feature. The secondary assessment may be of use later to refine the analysis process.

As a demonstration of the Radon profile visual classification process we select the example of the spiral galaxy 8979-12701. The Radon transform and Radon profile trace for this galaxy are shown in Figure \ref{fig:OB+IB}. Comparing the Radon trace profile with the model traces in Figure \ref{fig:class-models} and the examples given in  Figure 7 of \cite{2018MNRAS.480.2217S}, the visual classification process identified the strength of the features as: C = 1, IB = 2, OB = 2, IB+OB = 2+2 = 4, and A=0. This trace profile is comparable to the Type-OB+IB model and consequently the Radon profile of this galaxy is categorised as Type-OB+IB.

The classification process outlined above was carried out independently by 2 persons in an endeavour to provide some means of eliminating personal subjectivity. The intent is similar  to that adopted by the Galaxy Zoo project which used large-scale public collaboration to classify galaxy morphology \citep{2017MNRAS.464.4176W}. The galaxy-by-galaxy Radon profile Type classifications determined by the two assessors are listed in Tables \ref{tab:full-visual-classification} and \ref{tab:visual-classification-B} in Appendix \ref{sec:visual-classification-tables}.

\begin{figure}
    \centering
    \includegraphics[width=\columnwidth]{images/RadonPlots/RT-SNIPS-NEW/8979-12701-VA-OB+IB.png}
    \caption[Example of the Radon profile visual classification of galaxy 8979-12701]{Example of the Radon profile visual classification method for galaxy 8979-12701. The Radon transform (RT) plot is shown in the left panel and the Radon profile trace on the right. The RT plot minimum (magenta line) shows an indication of a wide outer bend (OB) feature. The trace plot also shows a narrower inner bend (IB) feature centred at radius $\rho=0$. We therefore classify the Radon profile of this galaxy as Type-OB+IB.}
    \label{fig:OB+IB}
\end{figure}

During the classification assignment exercise  some difficulties were encountered mainly with Radon trace profiles that did not fit easily into on of the 5 classification Types. An example of this is seen in the case of 8555-3701 where a clearly defined Inner Bend appears superimposed on an asymmetric trace as shown in Figure \ref{fig:8555-3701-A+IB}. The form of this trace does not fall readily into either of the Type-A or Type-IB categories, however faced with a choice of Types, and a requirement to select only one of the 5 categories, the natural choice was to opt for the predominant feature, in this case Type-A, asymmetric. The reader may disagree and opt for Type-IB, or even Type-OB+IB. This demonstrates the challenges encountered in the visual classification of Radon profiles.

\begin{figure}
    \centering
    \includegraphics[width=\columnwidth]{images/RadonPlots/RT-SNIPS-NEW/8555-3701-A+IB.png}
    \caption[Radon transform and profile trace plots for the galaxy 8555-3701]{Radon transform and profile trace plots for the galaxy 8555-3701. An inner bend appears superimposed on a generally asymmetric trace.}
    \label{fig:8555-3701-A+IB}
\end{figure}

In many other cases bends, or detectable velocity field disturbances, are evident as notches at well off-centre radii on otherwise constant or largely asymmetric traces. To obtain a comprehensive census at this level of detail these sub-dominant and off-centre features should be taken into account in a secondary analysis. Other than presenting a listing of the mixed secondary classifications obtained by classifier A in Table \ref{tab:full-visual-classification}, we do not pursued more detailed secondary analysis in this present work.




