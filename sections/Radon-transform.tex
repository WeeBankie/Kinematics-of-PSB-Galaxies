\section{Methods II: The Radon transform}
\label{sec:methods-II-Radon}

\subsection{Motivation}
\label{sec:motivation}

Many galaxies display kinematic features in their morphology such as warps, kinematically decoupled cores (KDCs), bars, oval distortions, tidal tails etc. The global kinematic $\Delta$PA${k}$ is single number value, representing a broad average for the the entire galaxy. Fine details of the radial variation across a velocity field can be obtained using the \textbf{Radon transform} method.

In addition, PSBs have depleted their gas and star formation has shut down. The differential kinematic position angles $\Delta$PA${k}$ cannot be easily measured in gas-poor galaxies like our PSBs. However the Radon transform method can be applied to either or both the gas and stellar velocity fields. For our purposes we apply the Radon transform to the stellar velocity maps (only) of gas-poor PSBs. 


\subsection{Description of the Radon transform}

The Radon transform (RT) is a mathematical transformation of an image devised to reveal internal properties of an object such as structure. The transform method was originally devised by Johann Radon in 1917 \citep{radon1917determination}. Since then the technique has been applied to many fields, including various branches of astronomy including microwave, radio and x-ray applications. \citet{deans2007radon} describes many of these applications. The book also includes an English translation of Radon's original German text. In another work the PhD thesis of  Peter Toft \citet{7910dc8d5b654c90ac4bc94c67d06f01} promotes the application of Radon transforms in the field of digital signal processing and presents algorithms for its implementation. For our present purposes \cite{2018MNRAS.480.2217S} describe the application of the Radon transform method to galaxy kinematic studies. They developed IDL code routines to analyse the kinematic properties of stellar and gas velocity fields obtained from the MaNGA integral field survey. The objective is to quantify radial variations in the kinematic position angles (PA$_{k}$).

\begin{figure}
    \centering
    \includegraphics[width=\columnwidth]{images/RadonPlots/Radon-transform-Stark.png}
    \caption[The Radon transform and its coordinate system]{The Radon transform: illustration of the Radon transform and its coordinate system taken from Figure 1 in \citet{2018MNRAS.480.2217S}. Line integrals across a 2-D image in x,y coordinate space are calculated along all possible lines, parameterised by the coordinates [$\theta$, $\rho$]. The left panel shows examples of two solid lines, L$_1$ and L$_2$, which are mapped through the Radon transform function to points $\theta_1$, $\rho_1$ and $\theta_2$, $\rho_2$ in [$\theta$, $\rho$] parameter space in the right-hand panel. Credit: Stark et al. 2018.}
    \label{fig:RadonTransform}
\end{figure}

Mathematically the Radon Transform, $R$, is defined as
\begin{equation}
    \label{eqn:radon}
    R(\rho,\theta)=\int_{L}{v(x,y)\, \diff l},
\end{equation}

where $v(x,y)$ is a 2D velocity field defined in Cartesian coordinates and $\int_{L}$ is the line integral at transform sky-plane polar coordinates ($\rho,\theta$). This transform is illustrated graphically in Figure \ref{fig:RadonTransform} where lines through an image in 2D velocity space are mapped to a series of points in Radon transform [$\theta,\rho$] space by means of line integrals at various angles about the velocity space axes, and at various radial distances offset from the velocity space origin. 


\citet{2018MNRAS.480.2217S} apply a modification to the Radon transform, to obtain the \textit{Absolute} Radon Transform, as defined in Equation (\ref{eqn:radon}), by taking the integral of the absolute values of the velocity field difference of each point $v(x_i,y_i)$ and the mean of all values along the line segment.

\begin{equation}
    \label{eqn:radon_absolute}
    R_A=\int{| v(x,y) - \langle v(x,y) \rangle | \, \diff l}.
\end{equation}

There is another variant to the Absolute Radon Transform described by \citet{2018MNRAS.480.2217S} named the \textit{bounded}, or aperture restricted, Radon transform, R\textsubscript{AB}. Briefly, R\textsubscript{AB} involves placing integration limits $(\pm{r_{ap}})$, known as the aperture, on the integral of Equation \ref{eqn:radon_absolute} in order to limit the number of spaxels across a velocity map that will be used in the integration. \citet{2018MNRAS.480.2217S} developed a set of IDL (Interactive Data Language) algorithms to calculate the Radon transform on an input array representing a velocity field. The Radon transform IDL code is available on GitHub\footnote{https://github.com/dvstark/radon-transform}. 
The graphical output of the Radon transform code as applied to a synthetic velocity field is shown in Figure \ref{fig:Radon}.

\begin{figure}
    \centering
   	\includegraphics[width=\columnwidth]{images/RadonPlots/example.png}
    \caption[Model velocity field Radon transform plots]{Example of Radon transform code output for a model velocity field obtained by running the Radon example code publicly available on David Stark's GitHub website. The left panel shows a synthetic uniform velocity field model, the middle panel shows the the absolute Radon transform of the velocity field and the right panel shows the aperture-restricted absolute transform. We are concerned with the latter, the absolute aperture-restricted Radon transform of stellar and gas velocity fields in this work. Credit: David Stark.}
    \label{fig:Radon}
\end{figure}

Following \citet{2018MNRAS.480.2217S} we demonstrate the application of the Radon transform to stellar and gas velocity field data obtained from the SDSS-IV MaNGA integral field survey for a selection of CPSB and RPSB galaxies. The Radon transform output for the stellar (left) and gas (right) velocity fields of the central PSB 8131-6101 is shown in Figure \ref{fig:RT_8131-6101}. High velocity regions are indicated in green and low velocity areas in purple. The stellar velocity Radon transform plot (left) reveals a well defined minimum track across the radial (vertical) coordinate, again shown in purple. Notably the right hand panel showing the gas velocity transform reveals few features, possibly due to a sparsity of gas in CPSB 8131-6101.

\begin{figure}
    \centering
   	\includegraphics[width=\columnwidth]{images/RadonPlots/RT-snips/CPSB-8313-6101-RT-snip.png}
    \caption[Example of basic Radon transform plots for gas and velocity fields of CPSB 8131-6101]{Radon transform plots of the stellar velocity (left panel) and gas velocity (right) fields of CPSB of MaNGA PLATEIFU 8131-6101.}
    \label{fig:RT_8131-6101}
\end{figure}

% 

\begin{figure}
    \centering
    \includegraphics[width=\columnwidth]{images/RadonPlots/RT-snips/CPSB-8313-6101-RT-snip.png}
    \includegraphics[width=\columnwidth]{images/RadonPlots/RT-snips/CPSB-9494-3701-snip.png}
    \includegraphics[width=\columnwidth]{images/RadonPlots/RT-snips/CPSB-8398-6102-snip.png}
    \caption{CPSBs: Radon transforms of stellar velocity and gas velocity maps. From the top CPSB-8313-6101, CPSB-9404-3710 and CPSB -8398-6103}
    \label{fig:CPSB-RTs}
\end{figure}   %% DO WE NEED THIS?

\subsection{Radon profile classification}
\label{sec:Radon-classification}

\cite{2018MNRAS.480.2217S} identified 5 commonly recurring patterns in the stellar and gas Radon profiles of their MaNGA sample. These patterns were used to classify the Radon profiles observed in their MaNGA sample. In this work we adopt the same classification approach for the Radon profiles of the PSB galaxies and their control samples. Simplified models of 4 of the classes are shown in Figure \ref{fig:class-models}. In addition an asymmetric profile class was identified. The salient features of the Radon profile classes are listed below:

\begin{itemize}
    \item Constant, \textbf{Type-C} : Radon profile with relatively constant trace minimum angle $\hat{\theta}$ at all radii $\rho$.
    \item Inner Bend, \textbf{Type-IB} : Galaxies whose Radon profiles exhibit symmetrical variations of $\hat{\theta}$ beginning at $|\rho|=0$, then transitioning to a constant value. 
    \item Outer Bend, \textbf{Type-OB} : Galaxies with constant Radon trace angle $\hat{\theta}$  at small $|\rho|$ which transition to a different value at a greater radius. 
    \item Inner Bend + Outer Bend, \textbf{Type-IB+OB} : Galaxies with Radon profiles showing a combination of the features of Type-IB and Type OB profiles.
    \item Asymmetric, \textbf{Type-A} : The value of the $\hat{\theta}$ varies significantly with $\rho$ across opposite sides of the transform R\textsubscript{AB}. 
 \end{itemize}

\begin{figure}
    \centering
    \includegraphics[width=\columnwidth]{images/RadonPlots/Radon-class-models.png}
    \caption{Toy models of the Radon profile (trace angle minimum $\hat\theta$ versus radius $\rho$). The sub-plots display examples of 4 of the Radon profile classes used for classification of the RT trace plots. Upper left: Constant, Type-C; upper right: Inner Bend, Type-IB; lower left: Outer Bend, Type-OB; and lower right: Inner Bend + Outer Bend, Type-IB+OB. Source: Stark et al. (2018) figure 8.}
    \label{fig:class-models}
\end{figure}

Mathematical functions describing these Radon profile classifications have been identified by \citet[][section 3.6]{2018MNRAS.480.2217S}. This has enabled code routines to be developed which can provide automatic classification of the Radon trace profiles for galaxies in MaNGA Product Launch MPL-5. However, these automatic classification routines are not presently available in the public domain. Instead, for this work, we adopt a simple visual classification method to categorise our sample galaxy into one of the 5 Radon profile trace types listed above. The visual classification process is described in the following 3-step process:

\begin{enumerate}
    \item Firstly we obtain the MAPS datacube FITS files for the selected PSB galaxies listed in Tables \ref{tab:my-CPSBs} and \ref{tab:my-RPSBs}, and a similar number of 'normal' galaxies drawn from their respective control samples, as described in Section \ref{sec:controls}, downloaded via the MaNGA Marvin web interface.
    \item Next, we process each of datacubes through the Radon transform wrapper code to obtain PostScript output files showing the galaxy SDSS $gri$ image cutout, the MaNGA stellar velocity map, the absolute bounded Radon transform R\textsubscript{AB} plot and the Radon profile plot of $\hat{\sigma}$ versus $\rho$. An example of this output is shown in Figure \ref{fig:RT-CPSB-9493-12705-SNIP}. 
    \item  Finally we examine the output plot for each galaxy and visually assess the relative qualitative strength of each of the 5 classification features by assigning a numeric weighting as given in Table \ref{tab:features}. This method adds a semi-quantitative approach to the visual assessment process.
\end{enumerate}

\begin{table}
    \centering
    \begin{tabular}{cl}
    \hline
    Value & Visual appearance \\
    \hline
    2 & The feature is visually predominant \\
    1 & Some evidence of the feature is apparent \\
    0 & The feature is absent \\
    \hline
    \end{tabular}
    \caption{Relative weighting of Radon profile feature strengths used in the  visual classification of Radon profile feature types: Constant, Inner Bend, Outer Bend, Inner Bend + Outer Bend or Asymmetric. The weighting is applied to quantify the visual appearance of the trace profile plots.}
    \label{tab:features}
\end{table}

In order to avoid a sample induced bias the galaxies are visually assessed in alphanumerical order of their PLATEIFU file name tag. No other information is used in visual assessment process. This is intended to avoid the pitfall of assessing one group, CPSBs, RPSBs, or their respective control sample groups, where common features may be apparent in a particular group. 

After inspecting the Radon transform and associated Radon trace plots for a particular galaxy an assessment of the strength of the Radon profile type features evident in the plots. A feature strength values from Table \ref{tab:features} is assigned to each of the 5 predefined Type classes for the galaxy. Based on the relative strength values allocated a predominant feature Type (C, IB, OB, IB+OB or A) is assigned to that particular galaxy. To determine the relative predominance of Type-IB+OB features we simply sum the strength values assigned to Types IB and OB together. 
In many cases there is some uncertainty in the absolute Type assessment, i.e. only one of the 5 defines classes, a secondary assessment is made,  generally this is the primary Type plus a less evident Type, or sub-dominant Type feature. The secondary assessment may be of use later in the analysis process. The complete Radon profile classification table, including the primary and secondary Type assessments, is provided in Appendix \ref{sec:visual-classification-tables}.

As an example of the classification process we can use the example of the gas velocity field of the (non-PSB) spiral galaxy 8313-12705 as shown in Figure \ref{gas-example}. Comparing the Radon trace profile with the model traces in Figure \ref{fig:class-models} the visual classification process yields: C=0, IB=1, OB=2, IB+OB=1+2=3, and A=0; consequently this galaxy is categorised as Type-OB+IB. 

During classification some difficulties were encountered mainly with Radon trace profiles that did not fit easily into on of the 5 classification Types. An example of this is clearly defined IB of 8555-3701 is imposed on an asymmetric trace as shown in Figure [TODO]. In many other cases bends, or velocity field disturbances, are evident as notches at well off-centre radii on otherwise constant or largely asymmetric traces. To obtain a comprehensive census at this level of detail these sub-dominant and off-centre, features should be into account in a secondary analysis, which is outside the scope of this present work.


The details of the visual classification for each galaxy is included [TODO] in Appendix \ref{sec:visual-classification-tables}.  






\input{sections/RT-plot-example.tex}    %% REFERENCE THIS SOMEWHERE



