% Abstract of the paper

Post-starburst (PSB) galaxies are identified from their spectra by the presence of strong Balmer series absorption lines and weak or absent emission lines. These features are consistent with a population of A-type and older stars that indicate star formation ceased within the past 1 to 2 Gyr. We analyse the kinematic properties of a sample of 68 PSB galaxies obtained from the Sloan Digital Sky Survey (SDSS) phase IV project: Mapping of Nearby Galaxies at Apache Point Observatory (MaNGA) integral field spectroscopic survey to ascertain if cessation of star formation in PSBs is consistent with major merger events. 

Major mergers can be revealed by asymmetries in the stellar and gas velocity fields. Firstly we look at differences in the kinematic position angles ($\Delta$PA$_{k}$) between the gas and stellar velocity fields for evidence of disruption caused by mergers. We then perform a Radon transform (RT) analysis on the stellar velocity field to identify radial variation and asymmetrical features, which could also indicate merger activity. 

We find that the $\Delta$PA$_{k}$ kinemetric analysis method when applied to PSB velocity fields is a good indicator of past merger activity, while the RT analysis method does not produce reliable results and requires further refinement. 