% Abstract of the paper

Post-starburst (PSB) galaxies can be identified from their spectra by the presence of strong Balmer series absorption lines and weak or absent emission lines. This is consistent with a population of A-type and older stars, indicating that star formation ceased within the past 1 to 2 Gyr. We analyse the kinematic properties of a sample of 68 PSB galaxies obtained from the Sloan Digital Sky Survey (SDSS) phase IV project: Mapping of Nearby Galaxies at Apache Point Observatory (MaNGA) integral field spectroscopic survey to determine if cessation of star formation in PSBs is consistent with evidence of major mergers. Major mergers may be revealed by asymmetries in the stellar and gas velocity fields. Firstly we look at differences in the kinematic position angles ($\Delta$PA$_{k}$) between the gas and stellar velocity fields for evidence of disruption. We then perform a Radon transform (RT) analysis to identify radial variation and asymmetry features present in the RT profile of PSB stellar velocity fields in an effort to identify underlying signatures of past merger activity.

The results of the kinematic position angle analysis show that PSBs exhibit a large range of $\Delta$PA$_{k}$ compared to a similar population of 'normal' or control galaxies. We conclude the $\Delta$PA$_{k}$ of PSBs show that they are a statistically different population from the control galaxies and this presents evidence of past disruption, possibly due to major mergers. The Radon transform analysis did not prove so conclusive however. Kinematic signatures of disruptive events consistent with major mergers are evident in a number of PSBs, however the data analysis technique will need to be refined in order to obtain more conclusive results. 


