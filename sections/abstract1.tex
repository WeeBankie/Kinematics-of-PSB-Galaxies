% Abstract of the paper

Post-starburst (PSB) galaxies can be identified from their spectra by the presence of strong Balmer series absorption lines and weak or absent emission lines. This is consistent with a population of A-type and older stars which is devoid of O- and B-type stars, indicating that star formation ceased within the past 1 to 2 Gyr. To investigate the possible causes of termination of star formation we employ kinematic analysis of stellar and gas velocity fields. Using data from the SDSS-IV MaNGA survey we study  kinematic properties of PSB galaxies in order to determine if star-formation quenching processes are consistent with major merger events which could trigger intense bursts of star formation and  consume much of the available gas. 

Major mergers may be revealed by asymmetries in the stellar and gas velocity fields. Firstly we look at differences in the kinematic position angles (PA$_{k}$) of the gas and stellar velocity fields determined from \texttt{kinemetry} analysis for evidence of major mergers. We then employ a Radon transform (RT) analysis to provide a finer level of detail in the kinematic analysis by identifying radial variation and asymmetry in the PA$_{k}$ of PSB stellar velocity fields. The Radon transform method has been already been applied to the broader SDSS-IV MaNGA survey in general. Here we apply the Radon transform technique specifically to the kinematic analysis of PSB galaxies in an effort to reveal underlying signatures of past merger activity that may have caused cessation of star formation, and possibly indicating morphological transition. The results show [TODO] ... 


