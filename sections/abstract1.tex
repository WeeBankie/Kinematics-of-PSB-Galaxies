% Abstract of the paper

Post-starburst (PSB) galaxies can be identified from their spectra by the presence of strong Balmer series absorption lines and weak or absent emission lines. This is consistent with a population of A-type stars and older indicating that star formation ceased within the past 1 to 2 Gyr. We use data from the SDSS-IV MaNGA survey to study the kinematic properties of PSB galaxies in order to determine if the star-formation quenching processes are caused by major mergers triggering a burst of intense star formation and consuming much of the available gas. Major mergers may be revealed by asymmetries in the stellar and gas velocity maps. Firstly we look at differences in the kinematic position angles (PA$_{k}$) of the velocity fields for evidence of major mergers. We then employ the Radon transform (RT) technique to provide a finer level of detail in the kinematic analysis by identifying radial variation and asymmetry in the PA$_{k}$ of PSB stellar velocity fields. The Radon transform method has been applied to the SDSS-IV MaNGA survey in general. Here we apply the technique specifically to the kinematic analysis of PSB galaxies.    


