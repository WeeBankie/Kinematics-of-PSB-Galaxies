% Abstract of the paper

The spectra of post-starburst (PSB) galaxies are typified by the presence of strong Balmer series absorption lines and weak or absent emission lines. These features are consistent with a population of A-type and older stars that indicate star formation ceased within the past 1 to 2 Gyr. We analyse the kinematic properties of a sample of 68 PSB galaxies obtained from the Sloan Digital Sky Survey (SDSS) phase IV project: Mapping of Nearby Galaxies at Apache Point Observatory (MaNGA) integral field spectroscopic survey to ascertain if the cessation of star formation in PSBs is consistent with past major mergers. 

Evidence of past major mergers can be revealed by asymmetries in the stellar and gas velocity fields. Firstly we look at differences in the kinematic position angles ($\Delta$PA$_{k}$) between the gas and stellar velocity fields of PSBs for evidence of disruption caused by mergers. We then performed a Radon transform (RT) analysis on the stellar velocity fields of PSBs to identify radial variation and asymmetrical features which could also indicate merger activity. 

We find that the $\Delta$PA$_{k}$ kinematic analysis method, together with supporting analyses, provides a good indicator of past merger activity. The Radon transform analysis, however, did not produce reliable results using our technique, which requires further refinement. 