\section{Discussion and conclusions}
\label{sec:discussion}

[TODO] Discuss the merits of the 3=person classification excercise.

[TODO] Discuss the Voronoi versus SPX binning method conparison.

[TODO] Kinemetry versus Radon transform. As described in Section \ref{sec:motivation} the Radon transform provides a distinct advantage over the kinemetry method in that it provides a detailed trace of the position angle across the velocity field enabling local regions of radial variation to be identified. 


\cite{2018MNRAS.480.2217S} made a considerable effort to explore a possible link between the observed frequencies of the 5 Radon profile types and underlying morphological features determined from the Galaxy Zoo morphology classification project. Their sample comprised the full $\sim$2500 galaxies from the SDSS MaNGA MPL-5 data release. The sample is representative of the entire population at z$\sim$0. They found evidence of weak associations between profile type and morphological features. These correlations are certainly not exclusive, and some care must be taken to follow their argument. The most prevalent of the identified relationships can be broadly interpreted as follows:
\begin{itemize}
    \item Type-C: Constant profile - frequently apparent in unbarred galaxies.
    \item Type-A: Asymmetric - can be associated with tidal interactions.
    \item Type-IB: Inner bends - prevalent in strongly barred galaxies.
    \item Type-OB: Outer bends - associated with kinematic warps.
\end{itemize}
Kinematic warps in the stellar discs of early-type field galaxies can be attributed to multiple stellar components suggestive of past dry mergers \citep{2005AJ....130.2647V}, although warps can also be sustained by tidal interactions.
From this we draw the simple conclusion that Radon trace profiles displaying outer bends, Type-OB or Type-OB+IB can be considered prospective candidates past major mergers, or post-mergers. From Table \ref{tab:Radon-VC-results} we find that 10 of of 27 CPSBs (37\%) reveal Type-OB or Type-OB+IB Radon profile signatures.   

[TODO] The Radon trace profiles for PSBs and their control groups are quantitatively similar in number. This is unexplained. We can expect more constant features in the RPSB groups than the CPSBs as the RPSBs have PSB features in local regions of these galaxies only while we can expect a higher percentage of non-linear features to be apparent in the trace profiles of CPSBs due to their global PSB region characteristics. 

[TODO] Expand this sentence to describe the prospective value of the SDSS imaging and MaNGA kinematic analysis in detecting merger signatures, in particular post-merger events as developed by \cite{2019DDA....5020304N}.