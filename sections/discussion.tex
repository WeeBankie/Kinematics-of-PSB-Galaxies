\section{Discussion and conclusions}
\label{sec:discussion}

Generally CPSBs show a wide range of kinematic position angle differences $\Delta$PA$_{k}$ with many in the range 90 \textless $\Delta$PA\_k \textless 180\textdegree. RPSBs show a smaller range in $\Delta$PA$_{k}$ with only a few over 90\textdegree. The majority of control galaxies have small kinematic position angle differences, mainly less than 25\textdegree. Significant misalignment in the gas and stellar velocity fields of CPSBs, and to a lesser extent RPSBs, is indicative of past or ongoing disturbances as could be expected from major mergers. 
We note that the distribution $\Delta$PA$_{k}$ is markedly different when comparing CPSBs with CPSB controls. This trait is also present, but to a lesser extent, in the comparison of the RPSB  $\Delta$PA$_{k}$ with their controls. The distribution in differential kinematic position angles is similar for both control groups. These observations are supported by the results of K-S analysis in Section \ref{sec:K-S-test}.

[TODO] Discuss the Voronoi versus SPX binning method comparison.

[TODO] Kinemetry versus Radon transform. As described in Section \ref{sec:motivation} the Radon transform provides a distinct advantage over the kinemetry method in that it provides a detailed trace of the position angle across the velocity field enabling local regions of radial variation to be identified. 

\cite{2018MNRAS.480.2217S} made a considerable effort to explore a possible link between the observed frequencies of the 5 Radon profile types and underlying morphological features determined from the Galaxy Zoo morphology classification project. Their sample comprised $\sim$2500 galaxies from the SDSS MaNGA MPL-5 data release. This sample is representative of the entire galaxy population at z$\sim$0. They found evidence of weak associations between profile type and morphological features. These correlations are certainly not exclusive, and some care must be taken to follow their argument. The most prevalent of the identified relationships can be broadly interpreted as follows:
\begin{itemize}
    \item Type-C: Constant profile - frequently apparent in unbarred galaxies.
    \item Type-A: Asymmetric - can be associated with tidal interactions.
    \item Type-IB: Inner bends - prevalent in strongly barred galaxies.
    \item Type-OB: Outer bends - associated with kinematic warps.
\end{itemize}
Kinematic warps in the stellar discs of early-type field galaxies can be attributed to multiple stellar components suggestive of past dry mergers \citep{2005AJ....130.2647V}, although warps can also be sustained by tidal interactions.
From this we draw the simple conclusion that Radon trace profiles displaying outer bends, Type-OB or Type-OB+IB can be considered prospective candidates past major mergers, or post-mergers. From Table \ref{tab:Radon-VC-results} we find that 10 of of 27 CPSBs (37\%) reveal Type-OB or Type-OB+IB Radon profile signatures. 
As an area for future study we should consider a similar exercise to utilise the Galaxy Zoo morphology classifications together with Radon profile features identified in this present study. 

Visual classification of Radon profile types was found to be very difficult; all our classifiers agreed on this point. In addition our classifiers arrived at varying interpretations of profile type class for most of the galaxies. Although the same written classification procedure was followed, the method was quite loosely defined, recognising the difficulties in relating the race plots to the examples. With so few classifiers involved we conclude that there is significant classification error present in our results, remembering the the Galaxy Zoo project engaged $\sim10^5$ classifiers to minimise classification error.

\cite{2018MNRAS.480.2217S} adopted a more quantitative approach employing numerical algorithms to fit the observed profiles to a standard profile model for each of the 5 profile type classes, such as a Gaussian fit for inner bends, and a fit to a Busy function \citep{2014ascl.soft02015W} for the outer bends. We would like to test our visual classifications against the results obtained from the automatic classification. 

[TODO include the results of the automatic classification] Their results were requested for this purpose, but due unforeseen circumstances, were not available for incorporation into our study.

The results of the visual classification of the 5 Radon trace type profiles for CPSBs, RPSBs and their control groups as presented in Table \ref{tab:Radon-VC-results} and shown graphically in Figure \ref{fig:Radon-grouped-barchart} are quantitatively similar in number. This was unexpected and is presently not fully explained. We can, however, expect more galaxies with constant features in the RPSB groups than the CPSBs as RPSBs exhibit PSB features in local regions only while we can expect a higher percentage of the non-linear features, i.e. inner and outer bends and composite bend features Type-OB+IB, to be apparent in the trace profiles of CPSBs due to their central and more widespread post-starburst regions. Incorporating the the results from the second classifier a picture emerges that Type-OB features are 70\% more prevalent in both the CPSB sample to controls, and RPSBs to their controls. The numbers in the samples employed are small and taken from MaNGA MPL-6 data. MPL-8 offers a much greater sample of PSBs and controls. A similar analysis as performed in this study would greatly enhance the statistical significance of this observation, possibly leading to a firm conclusion that PSB galaxies with Radon profile Type-OB signatures, may in fact reveal past major merger and coalescence evolutionary histories.

In the past major mergers have been detected using imaging techniques. In this project we have attempted to identify past major mergers in PSB galaxies, firstly by investigating the distributions of differences in stellar and gas velocity field kinematic position angles, and secondly by using the Radon transform method to reveal radial variation in kinematic position angles. This has not proved conclusive. More work is required on the classification of kinematic features apparent in the Radon profile trace plots. However, \cite{2019DDA....5020304N} have recently announced work on a method which promises to increase the accuracy of merger detection using a method that integrates imaging and kinematic analysis techniques. This improved method will combine SDSS imaging, providing morphological observations, and MaNGA  kinematic maps, towards the enhanced identification of merger signatures, in particular post-mergers. When available, this technique should be applied to our PSB and control samples to detect any signatures of past mergers. With this technique we may be able to increase our likelihood estimates of positive detection of past major merger events in post-starburst galaxies.