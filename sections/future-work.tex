\subsection{Future work}
\label{sec:future-work}
As with all studies our results and conclusions would benefit from an increased sample size. We focused on a total of 68 PSBs from MaNGA MPL-6 data. MPL-8 should encompass a much larger sample of PSBs. A similar analysis to that performed in this study using a larger sample of PSBs would improve the statistical significance of the results.

A larger sample from MPL-8  would present an opportunity to select a sub-sample of PSB based on inclination to exclude edge-on galaxies. While carrying out visual classification of Radon trace plots we noted that, in many cases, edge-on galaxies have poor quality stellar velocity maps which generated noisy Radon traces. These trace profiles were particularly difficult to classify by eye. Edge-on orientated galaxies can be selected by their low axis ratio, b/a. This data is available from the \texttt{DRPall} file in the \texttt{NSA\_ELPETRO\_BA} field. We recommend to revisit the visual classification of Radon profile features after taking a cut of PSBs and controls that have a larger b/a axis ratio, i.e. presenting a  more face-on aspect. This would exclude edge-on dusty galaxies. The cut of low inclination galaxies should provide a sub-population where merger signatures can be detected more readily.

We should incorporate the results produced by automated Radon trace classification. \cite{2018MNRAS.480.2217S} adopted a quantitative approach to Radon profile classification. Numerical algorithms were developed and fitted to each of the 5 profile type classes, for example a Gaussian fit for inner bends, and a fit to a Busy function \citep{2014ascl.soft02015W} for the outer bends. We planned to test the consistency of our visual classifications against the those obtained by the automatic classification. The automatic classification data for some of our galaxies became available late in the project. We intended to compare the classification results from the auto-classifier with our visual assessments but found that only 59 auto-classified galaxies (obtained from MaNGA MPL-5) were included within our sample of 127 PSBs and controls (from MPL-6). This detracted from the value of the comparison check exercise. In particular we noted that there was little consensus between classifications of either of classifier A, classifier B, or the auto-classifier. We recommend to investigate this disparity further, firstly by obtaining a full set of auto-classifications for our full sample of PSBs and controls, and then to investigate why there are inconsistencies in Radon profile type classifications by wither visual or automatic methods. 

As a further area for future study we should investigate any correlation between Galaxy Zoo morphology classifications and the Radon profile trace types as discussed this present study. It is envisaged that the catalogue of \citet{2018MNRAS.479..415A} will be useful here as it was specifically targeted at mergers. If a correlation is apparent it would enable us to refine the guidelines for our visual classification procedure, thereby resulting in more consistent results from diverse classifiers.