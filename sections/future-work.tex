\section{Future work}
\label{sec:future-work}

\begin{itemize}
    \item Quality screening cuts. While carrying out visual classification of Radon trace files it was noted in many cases that edge-on galaxies produced poor quality traces which were difficult to classify by eye. Edge-on orientated galaxies can be selected by their low axis ratio, b/a. This data is available from the \texttt{DRPall} file in the \texttt{NSA\_ELPETRO\_BA} field. We recommend to revisit the visual classification of Radon profile features taking a cut of PSBs and controls that have a larger b/a axis ratio, i.e. presenting a  more face-on sample with limiting inclination less than, say, 60\textdegree. This would exclude edge-on dusty galaxies in which merger signatures may be difficult to detect. The remaining face-on sample should provide a sub-population where merger signatures can be more readily detected.
    \item Automated Radon trace classification. \cite{2018MNRAS.480.2217S} adopted a more quantitative approach employing numerical algorithms to fit the observed profiles to a standard profile model for each of the 5 profile type classes, such as a Gaussian fit for inner bends, and a fit to a Busy function \citep{2014ascl.soft02015W} for the outer bends. We planned to test the consistency of our visual classifications against the those obtained by the automatic classification. The automatic classification data has become available recently. We were able to perform a simple comparison of the results of the auto-classifier to our visually assessed results. During this exercise we found  that only 59 auto-classified galaxies matched with our sample of 127 PSBs and controls, detracting from the significance of the analysis. In particular we noted that there are very few matches between classifications of either of classifier A, classifier B, or the auto-classifier. We recommend to investigate this disparity further, firstly to obtain a full auto-classification for our complete sample of PSBs and controls, then explore why there are differences in Radon profile type classification, automatic or visual. 
    \item The numbers in the samples employed are small and taken from MaNGA MPL-6 data. MPL-8 offers a much greater sample of PSBs and controls. A similar analysis to that performed in this study with a larger sample would greatly improve the statistical significance of the results. This could possibly lead to a tentative conclusion that PSB galaxies with Radon profile Type-OB signatures, may in fact reveal past major merger and coalescence evolutionary histories.
\end{itemize}