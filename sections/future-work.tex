\subsection{Future work}
\label{sec:future-work}
As with all studies our results and conclusion would benefit from an increased sample size. We focused on a total of 68 PSBs from MaNGA MPL-6 data. MPL-8 should encompass a much larger sample of PSBs. A similar analysis to that performed in this study using a larger sample of PSBs would improve the statistical significance of the results.

A larger sample from MPL-8  would present an opportunity to select a sub-sample of PSB based on inclination to exclude edge-on galaxies. While carrying out visual classification of Radon trace files it was noted in many cases that edge-on galaxies have poor quality stellar velocity maps and therefor generated noisy Radon traces. These were particularly difficult to classify by eye. Edge-on orientated galaxies can be selected by their low axis ratio, b/a. This data is available from the \texttt{DRPall} file in the \texttt{NSA\_ELPETRO\_BA} field. We recommend to revisit the visual classification of Radon profile features taking a cut of PSBs and controls that have a larger b/a axis ratio, i.e. presenting a  more face-on aspect. This would exclude edge-on dusty galaxies.. The remaining low inclination sample should provide a sub-population where merger signatures can be detected more readily.

Automated Radon trace classification. \cite{2018MNRAS.480.2217S} adopted a more quantitative approach employing numerical algorithms to fit the observed profiles to a standard profile model for each of the 5 profile type classes, such as a Gaussian fit for inner bends, and a fit to a Busy function \citep{2014ascl.soft02015W} for the outer bends. We planned to test the consistency of our visual classifications against the those obtained by the automatic classification. The automatic classification data has become available recently. We were able to perform a simple comparison of the results of the auto-classifier to our visually assessed results. During this exercise we found that only 59 auto-classified galaxies (obtained from MaNGA MPL-5) matched with our sample of 127 PSBs and controls (from MPL-6), detracting from the significance and value of the analysis. In particular we noted that there are very few matches between classifications of either of classifier A, classifier B, or the auto-classifier. We recommend to investigate this disparity further, firstly to obtain a full auto-classification for our complete sample of PSBs and controls, then explore why there are differences in Radon profile type classification, automatic or visual. 