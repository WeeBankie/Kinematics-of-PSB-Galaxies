\subsection{Intro guidance}
[JP input document intro-guidance.txt to be removed ]. Excellent start, thank you. Just to clarify the goal of this project is to use kinematic maps to ascertain if the post-starburst galaxies are caused by mergers (i.e. post-mergers). 

My apologies for the Swinbank reference. This one was the one I was thinking of: \citet{2012MNRAS.420..672S} One year out!
 
There are another couple of papers to look at:
Stark et al. 2018 : \citep{2018MNRAS.480.2217S} 
Barrera-Ballesteros et al. 2015 : \citet{2015A&A...582A..21B}.
 
Once you have looked at these, could you move up and down the references and cited papers in each paper, and see if you can find any other methods that have been used to identify merger or post-merger features \textbf{using kinematic features or maps}?
 
Extend your report to perhaps 1.5-2 pages to give a complete summary of the literature.
Remember that you are writing your report for a non-expert, so avoid jargon and explain symbols (i.e. K\_tot won't mean anything to the reader, or to me for that matter!). Whether a student can explain what they are doing to a non-expert is a key criteria for ascertaining whether they understood what they were doing, rather than just doing what their supervisor told them.  On discussion with Anne-Marie, we are not convinced that the full kinemetry fits will provide useful data on MaNGA galaxies. Note that it is important to get good marks on your final report that you provide a critical assessment of both your results and previous results, so have a think about the methods and what might work / not work on the MaNGA galaxies. 

\vspace{6pt}
\textbf{Remember to remove redundant subsection outlining placeholders.}