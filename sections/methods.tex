\section{Methods, Observations, Simulations etc.}
\label{sec:method} % used for referring to this section from elsewhere
Normally the next section describes the techniques the authors used.
It is frequently split into subsections, such as Section~\ref{sec:maths} below.
\subsection{Maths}
\label{sec:maths} % used for referring to this section from elsewhere
Simple mathematics can be inserted into the flow of the text e.g. $2\times3=6$
or $v=220$\,km\,s$^{-1}$, but more complicated expressions should be entered
as a numbered equation:
    \begin{equation}
        x=\frac{-b\pm\sqrt{b^2-4ac}}{2a}.
    	\label{eq:quadratic}
    \end{equation}
Refer back to them as e.g. equation~(\ref{eq:quadratic}). 

Note that this equation can appear centred using 
\verb'ceqn' or left-aligned using \verb'fleqn' in the \verb'\documentclass[]{}' statement.

\subsection{Figures and tables}

Figures and tables should be placed at logical positions in the text. Don't worry about the\textbf{ exact layout}, which will be handled by the publishers.

Figures are referred to as e.g. Fig.~\ref{}, and tables as
e.g. Table~\ref{tab:example_table}.


% Example table
\begin{table}
	\centering
	\caption{This is an example table. Captions appear above each table.
	Remember to define the quantities, symbols and units used.}
	\label{tab:example_table}
	\begin{tabular}{lccr} % four columns, alignment for each
		\hline
		A & B & C & D\\
		\hline
		1 & 2 & 3 & 4\\
		2 & 4 & 6 & 8\\
		3 & 5 & 7 & 9\\
		\hline
	\end{tabular}
\end{table}

