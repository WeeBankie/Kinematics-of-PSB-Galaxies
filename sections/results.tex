\section{Results}
\label{sec:results}

I have calculated a flux of 2.567 \fluxdensity.

The Sombrero galaxy lies at a distance of 10 \Mpc. It has a luminosity of about $10^{11}$ \Lsun, therefore having a mass of around $10^{11}$ \Msun.

Many other interesting results were obtained as reported in Table \ref{tab:resultstable}.

An additional finding was that the speed of light was determined to be approximately $c=3.10^8$ \ms.


% \begin{verbatim}
\begin{table}
 \centering
 \caption{Stellar masses and luminosities.}
  \label{tab:resultstable}
 \begin{tabular}{lcc}
  \hline
  Star & Mass & Luminosity\\
   & $M_{\sun}$ & $L_{\sun}$\\
  \hline
  Sun & 1.00 & 1.00\\
  $\alpha$~Cen~A & 1.10 & 1.52\\
  $\epsilon$~Eri & 0.82 & 0.34\\
  \hline
 \end{tabular}
\end{table}
% \end{verbatim}

\section{Post-starburst (PSB) Galaxies}
\label{sec:PSB_gals}
Vivienne has been involved in the preparation of a number of papers researching PSBs: \citet{2017MNRAS.472.1401A} regarding the relationship between quenching of star formation and morphological transition, while \citet{2016MNRAS.463..832W} sets out the background work.