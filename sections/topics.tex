\section{Topics}
Here is a list of ideas and topics to explore:
\begin{enumerate}
    \item Classification of PSBs: Post-starburst galaxies can be identified via the presence of prominent Hydrogen Balmer absorption lines in their spectra. 
    \item Diverse evolutionary channels of PSBs: see \citet{2019NatAs...3..440P}.
    \item \citet{2016ApJS..224...38A} : Shocked POststarbust Galaxy Survey. I. Candidate Post-starbust Galaxies with Emission Line Ratios Consistent with Shocks : There are many mechanisms by which galaxies can transform from blue, star-forming spirals, to red, quiescent early-type galaxies, but our current census of them does not form a complete picture. Recent observations of nearby case studies have identified a population of galaxies that quench “quietly.” 
    \item \citet{Mutch_2011} 5.3. Cold Gas Depletion in Spiral Galaxies? Due to their typically quiescent merger histories, it is more likely that in situ processes, rather than merger related processes, are responsible for stifling star formation in Milky Way like galaxies. In situ star formation in the model is directly driven by the amount of cold gas in, and the dynamical timescale of, the galactic disk (Kennicutt 1998). Simply put, the absence of cold gas implies the absence of star formation. Hence, we focus here on the evolving cold gas content of Milky Way like galaxies in our model.
    Cold gas in galaxies is regulated by five primary mechanisms: cooling flows from the halo, star formation, supernova feedback, AGN heating, and galaxy–galaxy mergers/interactions. The complex interplay between each is non-trivial to predict a priori, requiring the use of models and simulations.
\end{enumerate}